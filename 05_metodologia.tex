\section{Metodologia de Avaliação}
\label{Metodologia}

A metodologia de avaliação proposta para o presente estudo tem como objetivo avaliar o desempenho do Cluster K3s em ambientes de Internet das Coisas.
Para isso, será utilizado um modelo de aplicação que simula um sistema de monitoramento de sensores, onde os dados são enviados para um servidor central para processamento.
O modelo de aplicação foi pensado para representar um caso de uso comum de sistemas de Internet das Coisas, com processamento em forma de streaming de dados a partir dos clientes para o servidor.

Serão comparados cenários utilizando o Cluster K3s e o Kubernetes, e cenários sem a utilização de orquestradores de contêineres.
Ambos irão rodar em um mesmo ambiente de nuvem, com as mesmas configurações, para garantir a igualdade de condições de \textit{hardware}.
Serão avaliadas métricas gerais de consumo de recursos, como CPU, memória e rede, e métricas específicas de desempenho, como tempo de resposta e throughput para validação dos cenários

Serão realizados 4 cenários de avaliação, sendo eles:
\begin{itemize}
    \item \textbf{Cenário 1}: Ambiente sem orquestrador de contêineres com baixa carga de trabalho, buscando avaliar o consumo dos recursos sem a utilização de um orquestrador de contêineres controlando a distribuição de recursos.
    Nesse caso o contêiner terá disponível todos os recursos da máquina virtual.

    \item \textbf{Cenário 2}: Cluster K3s com baixa carga de trabalho, buscando avaliar o desempenho do Cluster K3s durante condições normais de operação.
    Nesse cenário busca-se entender o consumo de recursos base referente a utilização de um orquestrador de contêineres. 

    \item \textbf{Cenário 3}: Ambiente sem orquestrador de contêineres com alta carga de trabalho, buscando avaliar o desempenho de um ambiente sem orquestrador de contêineres durante condições de sobrecarga.
    Nesse cenário busca-se entender tanto o consumo de recursos quanto o throughput e qualidade da aplicação. 

    \item \textbf{Cenário 4}: Cluster K3s com alta carga de trabalho, buscando avaliar o desempenho do Cluster K3s durante condições de sobrecarga. 
    Busca-se também avaliar a capacidade de escalabilidade do Cluster K3s em um ambiente com baixa disponibilidade de recursos.

\end{itemize}
