\subsection{Funcionamento}

O modelo apresentado irá funcionar com base em 3 módulos em funcionamento na camada de Servidor apresentada na subseção \ref{Arquitetura}. 
Os 3 módulos buscam implementar um sistema de verificação e validação de dados vindos de diversos clientes, gerando notificações caso encontre algum problema. 
O recebimento das mensagens e processamento das mesmas será realizado a partir dos contêineres providos pelos Cluster K3s rodando no servidor, já o serviço de Notificação estará em funcionamento fora do Cluster K3s, como visto na Figura \ref{fig:Arq1}.

O modelo foi pensando para representar um caso de uso comum de sistemas de Internet das Coisas, com processamento em forma de streaming de dados a partir dos clientes para o servidor.
Como será utilizado o HiveMQ para a comunicação MQTT, o cliente poderá ser escalado de forma horizontal, aumentando o número de instâncias sem a necessidade de alterações no código fonte.
Desssa forma, será possíve aumentar significativamente o número de mensagens enviadas para o servidor sem a necessidade de alterações no código fonte, apenas com a adição de novas instâncias do cliente.
Como buscamos compreender o desempenho do Cluster K3s em ambientes de IoT, a camada do servidor contará apenas com escalabilidade horizontal dentro das capacidades da Máquina Virutal utilizada.

\begin{figure}[H]
    \centering
    \caption{Diagrama de sequência da aplicação}
    \includegraphics[width=1\linewidth]{images/sequence.png}\\
 Fonte: Elaborada pelo autor
    \label{fig:enter-label}
\end{figure}

\subsubsection{\textit{Message Handler}}

O \textit{Message Handler} será o módulo da aplicação responsável pela inscrição no tópico MQTT e o recebimento das mensagens. 
Esse módulo irá validar que a mensagem contém todos os campos obrigatórios para realização do processamento posterior. 
O \textit{Message Handler} não faz nenhum tipo de processamento e atua apenas como uma camada de conexão e recebimento de mensagens.
Caso a conexão com o brooker MQTT seja perdida, o \textit{Message Handler} irá tentar se reconectar ao brooker.
O \textit{Message Handler} e o \textit{Threshold Service} fazem parte da mesma aplicação, sendo assim, a comunicação entre eles será feita via chamadas de função.

Será criada uma configuração para o \textit{Message Handler} a partir de variáveis de ambiente que serão passadas no momento da criação do contêiner ou pela definição do deployment no Cluster K3s.
Essas variáveis irão especificar o endereço do brooker MQTT, o tópico que o \textit{Message Handler} irá se inscrever, assim como as credenciais de autenticação para se conectar com o brooker MQTT. 
Isso garante que o \textit{Message Handler} possa ser configurado de forma dinâmica e escalável, e que não exista conexões indesejadas com o sistema.

\subsubsection{\textit{Threshold Service}}

O \textit{Threshold Service} atuará como a camada de verificação de dados. 
Esse módulo irá receber as mensagens do \textit{Message Handler} e verificar se os valores recebidos estão dentro dos limites estabelecidos. 
Caso algum valor esteja fora do limite, o \textit{Threshold Service} irá se comunicar com o \textit{Notification Service} via uma chamada HTTP.
Essa chamada irá conter as informações necessárias para a identificação do cliente e do valor que está fora do limite.
Na figura TAL é possível visualizar a o formato de dados esperado pelo \textit{Notification Service}.

\subsubsection{\textit{Notification Service}}

O \textit{Notification Service} será o módulo responsável por enviar notificações para os clientes que tiveram valores fora do limite.
Como o \textit{Notification Service} é um serviço externo ao Cluster K3s, ele não terá acesso direto aos contêineres que estão rodando no servidor. 
Sendo assim, a comunicação entre o \textit{Threshold Service} e o \textit{Notification Service} será feita via chamadas HTTP.

Como o estudo em questão será focado na avaliação do desempenho do Cluster K3s, o \textit{Notification Service} será implementado de forma a apenas receber as chamadas HTTP e não enviar notificações reais para os clientes.
Ele será implementado para simular o custo das chamadas HTTP entre as aplicações e o tempo de resposta do serviço para criação de um cenário mais próximo do real.


